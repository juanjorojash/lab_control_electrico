\documentclass[aspectratio=169]{beamer}
\usetheme{Bruno}
\usepackage{amsmath}
\usepackage{amssymb}
\usepackage{siunitx}
\usepackage{float}
\usepackage{tikz}
\usepackage{url}
\usepackage[siunitx,american,RPvoltages]{circuitikz}
\ctikzset{capacitors/scale=0.7}
\ctikzset{diodes/scale=0.7}
\usepackage{tabularx}
\newcolumntype{C}{>{\centering\arraybackslash}X}
\renewcommand\tabularxcolumn[1]{m{#1}}% for vertical centering text in X column
\usepackage{tabu}
\usepackage[spanish,es-tabla,activeacute]{babel}
\usepackage{babelbib}
\usepackage{booktabs}
\usepackage{pgfplots}
\usepackage{hyperref}
\hypersetup{colorlinks = true,
            linkcolor = black,
            urlcolor  = blue,
            citecolor = blue,
            anchorcolor = blue}
\usepgfplotslibrary{units, fillbetween} 
\pgfplotsset{compat=1.16}
\usepackage{bm}
\usetikzlibrary{arrows, arrows.meta, shapes, 3d, perspective, positioning}
\renewcommand{\sin}{\sen} %change from sin to sen
\usepackage{bohr}
\setbohr{distribution-method = quantum,insert-missing = true}
\usepackage{elements}
\usepackage{verbatim}
\usetikzlibrary{mindmap,trees,backgrounds}
 
\definecolor{color_mate}{RGB}{255,255,128}
\definecolor{color_plas}{RGB}{255,128,255}
\definecolor{color_text}{RGB}{128,255,255}
\definecolor{color_petr}{RGB}{255,192,192}
\definecolor{color_made}{RGB}{192,255,192}
\definecolor{color_meta}{RGB}{192,192,255}
\usepackage[edges]{forest}
\usepackage{etoolbox}
\usepackage{schemata}
\newcommand\diagram[2]{\schema{\schemabox{#1}}{\schemabox{#2}}}
\title{Instrumentación I: \\ \emph{Introducción a la }\\ \emph{instrumentación industrial.} \\ \emph{Segunda sesión}}
\author{
    Juan J. Rojas, Hugo Sanchez Ortiz
}
\institute{Instituto Tecnológico de Costa Rica}
\date{\today}
\background{fig/background.jpg}
\begin{document}
\sisetup{unit-math-rm=\mathrm,math-rm=\mathrm} % change sinitx font
\sisetup{output-decimal-marker = {,}}
\maketitle

\newcommand{\blackandwhite}{white} %change this at the end

\begin{frame}{Rangos de entrada y sálida, sesgo de sensado}
    \begin{columns}[c, onlytextwidth]
        \begin{column}{0.50\textwidth}
            \begin{itemize}
                \item Rango del estímulo (span): representa la diferencia entre el valor mínimo y el valor máximo del estímulo que pueden medirse con un error aceptable.
                \item Salida a escala completa (full-scale output, FSO): representa la diferencia entre los valores de la señal de salida que corresponden al valor mínimo y máximo del estímulo medido. 
            \end{itemize}
        \end{column}
        \begin{column}{0.45\textwidth}
            \textbf{Ejemplo}\\[4pt]
            Marca: Amphenol\\[4pt]
            Modelo: NovaSensor NPI-19\\[4pt]
            Estímulo: presión\\[4pt]
            Tipo: piezoresistivo\\[4pt]
            Rango del estímulo: $0  - \SI{17.2}{\kilo\pascal}$\\[4pt]
            Salida a escala completa: $\SI{125}{\milli\volt}$\\[4pt]
            Sesgo de sensado: $\pm \SI{1}{\milli\volt}$\\[10pt]
            \tiny{Tomado de \href{https://f.hubspotusercontent40.net/hubfs/9035299/Product\%20Documents/AAS-920-298B-NovaSensor\%20NPI-19-REVISED-061714-web\%20(1).pdf}{aquí}}
        \end{column}
    \end{columns}
\end{frame}

\begin{frame}{Sesgo de sensado e histeresis}
    \begin{columns}[c, onlytextwidth]
        \begin{column}{0.50\textwidth}
            \begin{itemize}
                \item Sesgo de sensado (offset): es el rango de valores de la señal de salida cuando el estimulo tiene valor cero. 
                \item Histéresis: es una desviación de la señal de salida  que se obtiene al realizar la medición de un mismo valor de un estímulo en direcciones opuestas.
            \end{itemize}
        \end{column}
        \begin{column}{0.45\textwidth}
            \textbf{Ejemplo}\\[4pt]
            Marca: Allen-Bradley\\[4pt]
            Modelo: 875C AC\\[4pt]
            Estímulo: posición\\[4pt]
            Tipo: capacitivo\\[4pt]
            Histéresis: $\leq 10 \%$\\[10pt]
            \tiny{Tomado de \href{https://literature.rockwellautomation.com/idc/groups/literature/documents/td/875c-td001_-en-p.pdf}{aquí}}
        \end{column}
    \end{columns}
\end{frame}

\begin{frame}{Exactitud y precisión}
    \begin{columns}[c, onlytextwidth]
        \begin{column}{0.55\textwidth}
            \begin{itemize}
                \item Valor real de un estimulo: es el que se obtendría con una medición perfecta. Sin embargo este valor es indeterminado y por lo tanto se define por algún tipo de convención. 
                \item Exactitud: es la capacidad de un sensor de dar un resultado cercano al valor real del estímulo medido.
                \item Precisión: es la capacidad de un sensor de dar el mismo resultado cuando se mide el mismo valor de un estimulo bajo las mismas condiciones. 
            \end{itemize}
        \end{column}
        \begin{column}{0.40\textwidth}
            \textbf{Ejemplo}\\[4pt]
            Marca: Amphenol\\[4pt]
            Modelo: NovaSensor NPI-19\\[4pt]
            Estímulo: presión\\[4pt]
            Tipo: piezoresistivo\\[4pt]
            Exactitud estática$^*$: $0.1 \%$ \\[10pt]
            % Exactitud térmica del FSO: $\pm 0.5 \% $\\[4pt]
            % Exactitud térmica del offset: $\pm 0.5 \% $\\[10pt]
            \tiny{* Incluye errores de linealidad, histéresis y repetibilidad}\\
            \tiny{Tomado de \href{https://f.hubspotusercontent40.net/hubfs/9035299/Product\%20Documents/AAS-920-298B-NovaSensor\%20NPI-19-REVISED-061714-web\%20(1).pdf}{aquí}}
        \end{column}
    \end{columns}
\end{frame}

\begin{frame}{Repetibilidad y reproducibilidad}
    \begin{columns}[c, onlytextwidth]
        \begin{column}{0.60\textwidth}
            \begin{itemize}
                \item Repetibilidad: es una medida de la proximidad de la concordancia entre dos mediciones del mismo valor de un estímulo realizadas por la misma persona, con el mismo método e instrumento, bajo las mismas condiciones y en un periodo corto de tiempo. 
                \item Reproducibilidad: es una medida del grado de concordancia entre dos mediciones del mismo valor de un estímulo realizadas por diferentes personas, con el mismo método, con diferentes instrumentos y en un periodo largo de tiempo. 
            \end{itemize}
        \end{column}
        \begin{column}{0.35\textwidth}
            \includegraphics[width=0.9\textwidth]{fig/repeatibility.png}\cite{Fraden_2016}
            \begin{equation*}
                \delta_r = \dfrac{\Delta}{FS}\cdot 100\%
            \end{equation*}
        \end{column}
    \end{columns}
\end{frame}

\begin{frame}{Linealidad y resolución}
    \begin{columns}[c, onlytextwidth]
        \begin{column}{0.55\textwidth}
            \begin{itemize}
                \item Linealidad: es una medida de la proximidad entre la curva de calibración (valores medidos) y una linea recta específica.
                \item Resolución: es el incremento mas pequeño en el valor del estimulo que puede ser medido por el sensor. 
            \end{itemize}
        \end{column}
        \begin{column}{0.40\textwidth}
            \textbf{Ejemplo}\\[4pt]
            Marca: PIHER\\[4pt]
            Modelo: PS2P-LIN\\[4pt]
            Estímulo: posición\\[4pt]
            Tipo: magnetico, efecto Hall\\[4pt]
            Rango del estímulo: $\SI{12}{\milli\meter}$\\[4pt]
            Salida a escala completa (PWM): $10\% (\SI{-6}{\milli\meter}) \sim 90\% (\SI{6}{\milli\meter})$\\[4pt]
            Linealidad: $\pm 1 \%$ \\[4pt]
            Resolución(PWM): $14\,\mathrm{bits}$\\[10pt]
            \tiny{Tomado de \href{https://www.piher.net/pdf/PS2P-LINIntroduction.pdf}{aquí}}
        \end{column}
    \end{columns}
\end{frame}


\begin{frame}{Sensibilidad, saturación y banda muerta}
    \begin{columns}[c, onlytextwidth]
        \begin{column}{0.60\textwidth}
            \begin{itemize}
                \item Sensibilidad: es la pendiente de la curva de calibración, sin importar si es constante o no en todo el rango de salida.  
                \item Saturación: es el rango de operación en el que un incremento en el valor del estímulo no produce variaciones en la señal de salida del sensor. 
                \item Banda muerta: es el rango de operación en el que, sin importar el valor del estímulo, la salida es cercana a cero.
            \end{itemize}
        \end{column}
        \begin{column}{0.35\textwidth}
            \textbf{Ejemplo}\\[4pt]
            Marca: Allegro\\[4pt]
            Modelo: ACS723\\[4pt]
            Estímulo: corriente eléctrica\\[4pt]
            Tipo: magnetico, efecto Hall\\[4pt]
            Rango del estímulo: $\SI{10}{\ampere}$\\[4pt]
            Salida a escala completa: $\SI{0.5}{\volt}(\SI{-5}{\ampere}) \sim \SI{4.5}{\volt}(\SI{5}{\ampere})$\\[4pt]
            Sensibilidad: $\SI{400}{\milli\volt/\ampere}$ \\[4pt]
            Saturación: $<\SI{0.5}{\volt}, >\SI{4.5}{\volt}$\\[10pt]
            \tiny{Tomado de \href{https://www.allegromicro.com/-/media/files/datasheets/acs723-datasheet.ashx}{aquí}}
        \end{column}
    \end{columns}
\end{frame}

\begin{frame}{Valor real, error absoluto y relativo}
    \begin{columns}[c, onlytextwidth]
        \begin{column}{0.55\textwidth}
            \begin{itemize}
                \item Valor real: es el valor de salida que se obtendría con una medición perfecta. Debido a que esto es imposible de determinar se utiliza una convención para definir el valor real.  
                \item Error absoluto: es la diferencia entre una medición y el valor real. 
                \item Error relativo: es el cociente entre el error absoluto y el valor real
            \end{itemize}
        \end{column}
        \begin{column}{0.40\textwidth}
            \begin{equation*}
                \mathrm{error\,absoluto} = \mathrm{medición} - \mathrm{valor\,real}
            \end{equation*}
            \begin{equation*}
                \mathrm{error\,relativo} = \dfrac{\mathrm{error\,absoluto}}{\mathrm{valor\,real}}
            \end{equation*}
        \end{column}
    \end{columns}
\end{frame}

\begin{frame}{Referencias}
\bibliographystyle{ieeetr}
\footnotesize
\bibliography{comunes/referencias}
\end{frame}

\end{document}