\chapter{Laboratorio:  Arranques a tensión reducida y circuitos temporizados.}

\section{\obj}
\capacidad
\begin{itemize}
	{\small
	 \item Diseñar un circuito secuencial temporizado que permita arrancar un motor en tensión reducida utilizando el método de estrella delta.
    \item  Diseñar un circuito secuencial temporizado que permita arrancar un motor mediante auto-transformador.
    \item  Diseñar un circuito secuencial temporizado que permita arrancar en cascada dos motores y apagarlos en cascada. 
	 \item Alambrar el circuito de control y potencia.
	 \item Detectar fallas en circuitos alambrados con lógica cableada.
	 
 }
\end{itemize} 

 
\section{Equipos y materiales}
Para este laboratorio de necesitaran:
\begin{itemize}
	{\small \item 2 motor trifásico 1/2 hp de 3 puntas.
	\item 1 motor trifásico 1/2 hp de 6 puntas.
	\item 2 relevadores térmico.
	\item 4 contactores.
	\item 1 botón pulsar cerrado.
	\item 2 botones pulsadores abiertos.
	\item 1 temporizador On-Delay.
	\item 1 temporizador Off-Delay.
	\item 1 amperimetro de gancho con medición INRUSH.
	\item 1 Analizador de señales Fluke 143B
}
\end{itemize}

\subsection{Arranque a tensión reducida}

Los arranques a tensión reducida buscan reducir el voltaje que alimenta los devanados estatóricos para que en el momento del arranque la corriente de línea se reduzca.

El arranque por autotranformador, utilizado en motores trifásicos de inducción jaula de árdilla de tres terminales, permite reducir tanto el voltaje de línea como la corriente de linea que consume el motor. Si la relacion de tranformacion es $a=V_{out}/V_{in}$, donde $V_{out}$ es el voltaje hacia el motor y $V_{in}$ es el voltage de la red. La corriente de línea antes del autotransformador es $I_{linea} = a^{2}\cdot I_{arr}$, donde $I_{arr}$ es la corriente de arranque a tensión plena. La Figura \ref{fig:control-auto-transformador} muestra el circuito de potencia y control para el arranque por auto-transformador.

Los motores trifásicos de indución jaula de ardilla de seis puntas se pueden arrancar realizando una reconfiguración de su devanado. Durante el régimen transitorio  sus devanados se alambran en estrella ($\leftthreetimes$) y cuando su velocidad alcanza el 80\% aproximadamente se realiza la re-conexión en delta ( $\Delta$) de los devanados. Esto permiten reducir el voltaje en $\sqrt{3}$ y la corriente de arranque se reduce a un tercio de la original. La Figura \ref{fig:potencia-estrella-delta} muestra el circuito de potencia para el arranque $\leftthreetimes-\Delta$.

\section{Metodología}

Este laboratorio tiene una duración de 4 lecciones, repartidas en dos semanas. Los estudiantes deben mostrar durante las clases programadas las tres actividades propuestas. Deben recabar fotografías y resultados de los equipos de medición para elaborar las evidencias. Las evidencias se subirán al TecDigital la semana siguiente finalizadas las actividades.

\section{Práctica en Clase}

\subsection{Actividad 1}

Realice un arranque-pare secuencial para dos motores trifásicos de tres puntas. Mida la corriente de arranque y la tensión de alimentación del motor.

\begin{figure}[H]
\centering
    \begin{tikzpicture}[circuit plc ladder,scale=1.3]
    \draw(0,0)  
    to [ncpb,l=$P$] ++ (1,0) coordinate(A1) 
    to [nopb,l=$A$] ++ (1,0) coordinate(A2) 
    to [esource, l=$M_1$] ++ (1,0) coordinate(A3) 
    to [contact NC={info={$S_{c1}$}}] ++ (1,0) 
    to [contact NC={info={$S_{c2}$}}] ++ (1,0) coordinate(laddertopright)
    (A1) -- ++ (0,-0.5) 
    to [contact NO={info={$M_1$}}] ++ (1,0) -- (A2)
    (A2) -- ++ (0,-1)
    to [esource, l=$T_{on}$] ++ (1,0) -- ++ (0,1)
    (A2) -- ++ (0,-2)
    to [esource, l=$T_{off}$] ++ (1,0) -- ++ (0,1)
    (0,-3) -- ++ (1,0)
    to [contact NO={info={$T_{on}$}}] ++ (1,0)
    to [esource, l=$M_2$] ++ (1,0) -- ++ (0,1)
    (0,-4) 
    to [contact NO={info={$M_2$}}] ++ (1,0)
    to [contact NO={info={$T_{off}$}}] ++ (1,0) -- ++ (0,1)
    ;
    \ladderrungend{5}
    \ladderpowerrails
    \end{tikzpicture}
    \caption{Circuito de control: arranque-pare secuencial de dos motores}
    \label{fig:control-secuencial}
\end{figure}

\begin{figure}[H]
\centering
    \begin{circuitikz}[cute inductors,scale=0.7]
    \node[esourceshape,name=motor1] at (8,-2){};
    \node[esourceshape,name=motor2] at (8,-8){};
    \draw
    (0,0) node[left]{$L_1$} 
    --++ (1.5,0) coordinate(L11) -- ++ (0.5,0)
    to [C,l=$M_1$] 
    ++ (2,0) to [wfuse,l=$S_{c1}$] ++ (2,0) -- (motor1.120)
    (0,-2) node[left]{$L_2$} 
    --++ (1,0) coordinate(L21) -- ++ (1,0)
    to [C,l=$M_1$] 
    ++ (2,0) to [wfuse,l=$S_{c1}$] ++ (2,0) -- (motor1.180)
    (0,-4) node[left]{$L_3$} 
    --++ (0.5,0) coordinate(L31) -- ++ (1.5,0)
    to [C,l=$M_1$] 
    ++ (2,0) to [wfuse,l=$S_{c1}$] ++ (2,0) -- (motor1.240)
    (L11) 
    to[short,*-] 
    ++ (0,-6) -- ++ (0.5,0) 
    to [C,l=$M_2$] 
    ++ (2,0) to [wfuse,l=$S_{c2}$] ++ (2,0) -- (motor2.120)
    (L21) 
    to[short,*-] 
    ++ (0,-6) -- ++ (1,0) 
    to [C,l=$M_2$] 
    ++ (2,0) to [wfuse,l=$S_{c2}$] ++ (2,0) -- (motor2.180)
    (L31) 
    to[short,*-] 
    ++ (0,-6) -- ++ (1.5,0) 
    to [C,l=$M_2$] 
    ++ (2,0) to [wfuse,l=$S_{c2}$] ++ (2,0) -- (motor2.240)
    (motor1.center) node{$M$}
    (motor2.center) node{$M$}
    ; 
    
    \end{circuitikz}
    \caption{Circuito de potencia: arranque-pare secuencial de dos motores}
    \label{fig:potencia-secuencial}
\end{figure}


 
\subsubsection{Conteste las preguntas}

¿Puede mostrar el circuito de control para ambas variantes y las ecuaciones lógicas?
¿Puede mostrar el diagrama de tiempos (oscilograma) del funcionamiento de cada bobina, y su relación con los eventos?
¿El circuito implementado funciona según lo planeado? ¿Cuanto es la corriente de arranque medida con el FLUKE 143B?

\subsection{Actividad 2}
 
 Realice un arranque por auto-transformador para un motor trifásico de tres puntas. Mida la corriente de arranque  y la tensión de alimentación del motor.

\begin{figure}[H]
\centering
    \begin{circuitikz}[cute inductors]
    \node[esourceshape,name=motor] at (7,-4){};
    \draw
    (0,0) node[left]{$L_1$}
    to [C,l=$G$] ++ (2,0) -- ++ (1,0) coordinate(top)
    (0,-4) node[left]{$L_2$}
    to [C,l=$G$] ++ (2,0) -- ++ (1,0) coordinate(mid)
    (0,-8) node[left]{$L_3$}
    to [C,l=$G$] ++ (2,0) -- ++ (1,0) coordinate(bot)
    (top)
    to [L,name=L1] ++ (0,-3)
    to [C,l=$E$] (mid)
    to [C,l=$E$] ++ (0,-1)
    to [L,name=L2] ++ (0,-3)
    (L1.midtap) -- ++ (1,0) 
    to [C,l_=$D$] ++ (0,1.5) -- (top)   
    (L2.midtap) -- ++ (1,0) 
    to [C,l_=$D$] ++ (0,-1.5) -- (bot) 
    (L1.midtap) node[below,xshift=22]{$H_3 H_2$} ++ (1,0) to [wfuse,l=$S_{c}$] ++ (2,0) -- (motor.120)
    (L2.midtap) node[above,xshift=22]{$H_3 H_2$} ++ (1,0) to [wfuse,l=$S_{c}$] ++ (2,0) -- (motor.240)
    (mid) --++ (1,0) to [wfuse,l=$S_{c}$] ++ (2,0) -- (motor.180)
    (motor.center) node{$M$}
    (L1.core west) node[left,xshift=-8]{$H_1$}
    (L1.core east) node[left,xshift=-8]{$H_5$}
    (L2.core west) node[left,xshift=-8]{$H_1$}
    (L2.core east) node[left,xshift=-8]{$H_5$}
    ;
    \end{circuitikz}
    \caption{Circuito de potencia: arranque por auto-transformador}
    \label{fig:potencia-auto-transformador}
\end{figure}

\begin{figure}[H]
\centering
    \begin{tikzpicture}[circuit plc ladder,scale=1.3]
    \draw(0,0)  
    to [ncpb,l=$P$] ++ (1,0) coordinate(A1)
    to [nopb,l=$A$] ++ (1,0) coordinate(A2)
    to [contact NC={info={$Sc$}}] ++ (1,0)
    to [esource, l=$G$] ++ (1,0) coordinate(laddertopright)
    (A1) -- ++ (0,-0.5) 
    to [contact NO={info={$G$}}] ++ (1,0) -- (A2)
    ;
    \ladderrungend{1}
    \draw(0,0)
    to [contact NO={info={$G$}}] ++ (1,0) coordinate(T1) --++ (2,0)
    to [esource, l=$T_{on}$] ++ (1,0)
    (T1) -- ++ (0,-1)
    to [contact NC={info={$T_{on}$}}] ++ (1,0)
    to [contact NC={info={$D$}}] ++ (1,0)
    to [esource, l=$E$] ++ (1,0)
    (T1) ++ (0,-1) --++ (0,-1)
    to [contact NO={info={$T_{on}$}}] ++ (1,0)
    to [contact NC={info={$E$}}] ++ (1,0)
    to [esource, l=$D$] ++ (1,0)
    ;
    \ladderrungend{3}
    \ladderpowerrails
    \end{tikzpicture}
    \caption{Circuito de control: arranque por auto-transformador}
    \label{fig:control-auto-transformador}
\end{figure}

 
 
\subsubsection{Conteste las preguntas:}

¿Cual es la corriente de arranque del motor según su dato de placa y letra de código? ¿Cuanto es la corriente de arranque medida con el FLUKE 143B? ¿Cuanto se redujo la corriente en el arranque?¿Se cumple la ecuación $I_{linea} = a^{2}\cdot I_{arr}$ ?

\subsection{Actividad 3}

Realice un arranque pare reversible para un motor que arranca en estrella-delta. Es decir, el motor debe arrancar en estrella-delta hacia la derecha, o puede arrancar en estrella-delta hacia la izquierda. Cuando se presiona el botón de pare o sobrecarga este se detiene. Solo se requiere un temporizador tipo On-Delay.

\begin{figure}[H]
\centering
    \begin{circuitikz}[cute inductors,scale=0.7]
    \draw
    (0,0) node[left]{$L_1$} 
    --++ (1.5,0) coordinate(L11) -- ++ (0.5,0)
    to [C,l=$F$] 
    ++ (2,0) -- ++ (1.5,0) coordinate(L12) -- ++ (0.5,0) coordinate(L13)
    (0,-2) node[left]{$L_2$} 
    --++ (1,0) coordinate(L21) -- ++ (1,0)
    to [C,l=$F$] 
    ++ (2,0) -- ++ (1,0) coordinate(L22) -- ++ (1,0) coordinate(L23) 
    (0,-4) node[left]{$L_3$} 
    --++ (0.5,0) coordinate(L31) -- ++ (1.5,0)
    to [C,l=$F$] 
    ++ (2,0) -- ++ (0.5,0) coordinate(L32) -- ++ (1.5,0) coordinate(L33)
    (L31) 
    to[short,*-] 
    ++ (0,6) -- ++ (1.5,0) 
    to [C,l=$R$] 
    ++ (2,0) -- ++ (0.5,0) 
    to[short,-*]
    (L32)
    (L21) 
    to[short,*-] 
    ++ (0,6) -- ++ (1,0) 
    to [C,l=$R$] 
    ++ (2,0) -- ++ (1.5,0) 
    to[short,-*]
    (L12)
    (L11) 
    to[short,*-] 
    ++ (0,6) -- ++ (0.5,0) 
    to [C,l=$R$] 
    ++ (2,0) -- ++ (1,0) 
    to[short,-*]
    (L22)
    (L13) to[wfuse,l=$S_{c}$] ++ (2,0) --++ (4,0) node[above]{$6$}
    to [L,name=L1,*-*]
    ++ (2,0) node[above]{$3$} -- ++ (3.5,0) coordinate(L14) -- ++ (0.5,0) coordinate(L15)
    (L23) to[wfuse,l=$S_{c}$] ++ (2,0) --++ (4,0) node[above]{$5$}
    to [L,name=L2,*-*]
    ++ (2,0) node[above]{$2$} -- ++ (3,0) coordinate(L24) -- ++ (1,0) coordinate(L25)
    (L23) ++ (7,0) node[yshift=-75]{motor} circle (3)
    (L33) to[wfuse,l=$S_{c}$] ++ (2,0) --++ (4,0) node[above]{$4$}
    to [L,name=L3,*-*]
    ++ (2,0) node[above]{$1$} -- ++ (2.5,0) coordinate(L34) -- ++ (1.5,0) coordinate(L35)
    (L32) 
    ++ (4,0) 
    to[short,*-]
    ++ (0,6) -- ++ (3.5,0)
    to[C,l=$D$]
    ++ (2,0) -- ++ (3,0)
    to[short,-*] 
    (L24)
    (L22) 
    ++ (4,0) 
    to[short,*-]
    ++ (0,6) -- ++ (3,0)
    to[C,l=$D$]
    ++ (2,0) -- ++ (3.5,0)
    to[short,-*] 
    (L14)
    (L12) 
    ++ (4,0) 
    to[short,*-]
    ++ (0,6) -- ++ (2.5,0)
    to[C,l=$D$]
    ++ (2,0) -- ++ (2.5,0)
    to[short,-*] 
    (L34)
    (L15)
    to[C,l=$E$] 
    ++ (2,0) -- ++ (1,0)
    to[short,-*] ++ (0,-2)
    (L25)
    to[C,l=$E$] 
    ++ (2,0) -- ++ (1,0)
    (L35)
    to[C,l=$E$] 
    ++ (2,0) -- ++ (1,0)
    to[short,-*] ++ (0,2)
    ;
    \end{circuitikz}
    \caption{Circuito de potencia: arranque Estrella-Delta Reversible}
    \label{fig:potencia-estrella-delta}
\end{figure}

\begin{figure}[H]
\centering
    \begin{tikzpicture}[circuit plc ladder,scale=1.3]
    \draw(0,0)  
    to [ncpb,l=$P$] ++ (1,0) coordinate(AF1)
    to [nopb,l=$A_F$] ++ (1,0) coordinate(AF2)
    to [contact NC={info={$R$}}] ++ (1,0)
    to [esource, l=$F$] ++ (1,0) 
    to [contact NC={info={$S_c$}}] ++ (1,0) coordinate(laddertopright)
    (AF1) -- ++ (0,-0.5) 
    to [contact NO={info={$F$}}] ++ (1,0) -- (AF2)
    (AF1) -- ++ (0,-1.5) coordinate(AR1)
    to [nopb,l=$A_R$] ++ (1,0) coordinate(AR2)
    to [contact NC={info={$F$}}] ++ (1,0)
    to [esource, l=$R$] ++ (1,0) -- ++ (0,1.5)
    (AR1) -- ++ (0,-0.5) 
    to [contact NO={info={$R$}}] ++ (1,0) -- (AR2)
    (0,-4)
    to [contact NO={info={$R$}}] ++ (1,0)
    (0,-3)
    to [contact NO={info={$F$}}] ++ (1,0) coordinate(FR) -- ++ (2,0)
    to [esource, l=$T_{on}$] ++ (1,0) -- ++ (0,1.5)
    (FR) -- ++ (0,-1) 
    to [contact NC={info={$T_{on}$}}] ++ (1,0)
    to [contact NC={info={$D$}}] ++ (1,0)
    to [esource, l=$E$] ++ (1,0)
    (FR) ++ (0,-1) --++ (0,-1)
    to [contact NO={info={$T_{on}$}}] ++ (1,0)
    to [contact NC={info={$E$}}] ++ (1,0)
    to [esource, l=$D$] ++ (1,0) -- ++ (0,2)   
    ;
    \ladderrungend{6}
    \ladderpowerrails
    \end{tikzpicture}
    \caption{Circuito de control: arranque Estrella-Delta Reversible}
    \label{fig:control-estrella-delta}
\end{figure}

\subsubsection{Conteste las preguntas:} 

 ¿Como es el circuito implementado? ¿Cuanto es la corriente de arranque medida con el FLUKE 143B? ¿Como es el diagrama de tiempos de cada bobina? ¿La corriente de arranque se redujo un tercio del arranque a plena tensión? 



