\chapter{Repositorio de código}
\label{ap:osc}
\section{Código para sección \ref{l2:a1}}
\label{ApendiceA}

{\scriptsize 
    \begin{lstlisting}[language=Arduino,numbers=none, showstringspaces=false]
    /*
      ASCII table
    
      Prints out byte values in all possible formats:
      - as raw binary values
      - as ASCII-encoded decimal, hex, octal, and binary values
    
      For more on ASCII, see http://www.asciitable.com and http://en.wikipedia.org/wiki/ASCII
    
      The circuit: No external hardware needed.
    
      created 2006
      by Nicholas Zambetti <http://www.zambetti.com>
      modified 9 Apr 2012
      by Tom Igoe
    
      This example code is in the public domain.
    
      https://www.arduino.cc/en/Tutorial/BuiltInExamples/ASCIITable
    */
    
    void setup() {
      //Initialize serial and wait for port to open:
      Serial.begin(9600);
      while (!Serial) {
        ;  // wait for serial port to connect. Needed for native USB port only
      }
    
      // prints title with ending line break
      Serial.println("ASCII Table ~ Character Map");
    }
    
    // first visible ASCIIcharacter '!' is number 33:
    int thisByte = 33;
    // you can also write ASCII characters in single quotes.
    // for example, '!' is the same as 33, so you could also use this:
    // int thisByte = '!';
    
    void loop() {
      // prints value unaltered, i.e. the raw binary version of the byte.
      // The Serial Monitor interprets all bytes as ASCII, so 33, the first number,
      // will show up as '!'
      Serial.write(thisByte);
    
      Serial.print(", dec: ");
      // prints value as string as an ASCII-encoded decimal (base 10).
      // Decimal is the default format for Serial.print() and Serial.println(),
      // so no modifier is needed:
      Serial.print(thisByte);
      // But you can declare the modifier for decimal if you want to.
      // this also works if you uncomment it:
    
      // Serial.print(thisByte, DEC);
    
    
      Serial.print(", hex: ");
      // prints value as string in hexadecimal (base 16):
      Serial.print(thisByte, HEX);
    
      Serial.print(", oct: ");
      // prints value as string in octal (base 8);
      Serial.print(thisByte, OCT);
    
      Serial.print(", bin: ");
      // prints value as string in binary (base 2) also prints ending line break:
      Serial.println(thisByte, BIN);
    
      // if printed last visible character '~' or 126, stop:
      if (thisByte == 126) {  // you could also use if (thisByte == '~') {
        // This loop loops forever and does nothing
        while (true) {
          continue;
        }
      }
      // go on to the next character
      thisByte++;
    }
    \end{lstlisting}
}



\section{Código actividad 2}
\label{ApendiceB}
{\scriptsize 
    \begin{lstlisting}[language=Arduino,numbers=none, showstringspaces=false]
    /*
    Instituto Tecnológico de Costa Rica
    LAB #1 de Laboratorio de Control Eléctrico.
    Fecha: 25/01/2023
    Ing. Luis D. Murillo
    
    Este programa apaga el LED cuando se presiona el botón. 
    
    */
    // Declaracione de entradas y salidas
    const int BUTTON =2;  // Boton conectado al pin 2
    const int LED =4;     // LED conectado al pin 4
    
    // CONFIGURACION DE ENTRADAS Y SALIDAS
    void setup() {
        pinMode(LED, OUTPUT);    // configuramos el pin del LED como salida
        digitalWrite(LED, HIGH);  // Encendemos el LED
        pinMode(BUTTON, INPUT);  // configuramos el pin del botón como entrada
    }
    // PROGRAMA PRINCIPAL
    void loop() {
        bool buttonState = digitalRead(BUTTON);  // leemos el estado del botón
        if (buttonState == HIGH) {
            digitalWrite(LED, LOW);  // apagamos el LED
        } else {
            digitalWrite(LED, HIGH);   // encendemos el LED
        }
    }
    \end{lstlisting}
}	


\section{Código actividad 3}

{\scriptsize 
		
	\begin{lstlisting}[language=Arduino,numbers=none, showstringspaces=false]
		 /*
		Instituto Tecnológico de Costa Rica
		Laboratorio de Control Eléctrico.
		Lab #1: Conectiva logicas y señal analógica
		Fecha: 24/01/2023
		Ing. Luis D. Murillo
		
		Implementación de Funciones Lógicas AND, OR, XOR, NAND, NOR de 2 entradas
		*/
		// Declaracione de constantes
			const int PinEntrada[2]={2,3};
			const int PinSalidas[5]={4,5,6,7,8};
			const int PinAnalogico[2] = {A0, A1};
			int ValorAnalojLeido[2]={0,0};
			float ValorVoltage[2]={0.0,0.0};
			boolean Mapa_entradas[2];
			boolean ResultadoLogico[5]={false, false, false, false, false};
		
		// Configuracion de Pines de entrada y salida
		void setup(){
			Serial.begin(9600);
			Serial.println("-----");
			Serial.println((sizeof(PinEntrada)/2));
			Serial.println((sizeof(PinSalidas)/2));
			Serial.println("-----");
		
		// Initializa los pines:
		for (int i = 0; i < (sizeof(PinSalidas)/2); i++) {
			if (i<(sizeof(PinEntrada)/2)) {
				pinMode( PinEntrada[i], INPUT);
			}
			pinMode(PinSalidas[i], OUTPUT);
			}
		delay(2);
		}
		
		void loop()
		{		
			//LECTURA DE LAS ENTRADAS Y SALIDAS 	
			for(int i=0; i<(sizeof(PinSalidas)/2);i++){
				if(i<(sizeof(PinEntrada)/2)){
					// Lectura de entradas digitales
					Mapa_entradas[i]=digitalRead(PinEntrada[i]);
					// Lectura de los valores analógicos
					ValorAnalojLeido[i] = analogRead(PinAnalogico[i]);
					// Funcion de mapeo :
				ValorVoltage[i]= (map(ValorAnalojLeido[i], 0, 1023, 0, 500)/100.0);
				}
			// Escritura de salida digital
			digitalWrite(PinSalidas[i],ResultadoLogico[i]);
			}
		
		//EJECUCION DEL PROGRAMA
			ResultadoLogico[0]=AND(Mapa_entradas[1],Mapa_entradas[0]);
			ResultadoLogico[1]=OR(Mapa_entradas[1],Mapa_entradas[0]);
			ResultadoLogico[2]=XOR(Mapa_entradas[1],Mapa_entradas[0]);
			ResultadoLogico[3]=NAND(Mapa_entradas[1],Mapa_entradas[0]);
			ResultadoLogico[4]=NOR(Mapa_entradas[1],Mapa_entradas[0]);
		
		//IMPRESION DE RESULTADOS ENTRADA Y SALIDAS
		
			Serial.print(Mapa_entradas[0]);
			Serial.print(", ");
			Serial.print(ValorVoltage[0]);
			Serial.print(", ");
			Serial.print(Mapa_entradas[1]);
			Serial.print(", ");
			Serial.print(ValorVoltage[1]);
			Serial.print(", ");
			Serial.println(ResultadoLogico[0]);
			Serial.print(", ");
			Serial.print(ResultadoLogico[1]);
			Serial.print(", ");
			Serial.print(ResultadoLogico[2]);
			Serial.print(", ");
			Serial.print(ResultadoLogico[3]);
			Serial.print(", ");
			Serial.println(ResultadoLogico[4]);
		}
		
		//DEFINICION DE LAS FUNCIONES LÓGICAS
		// Forma de programacion Booleana
		bool AND (bool X, bool Y ){
		return (X && Y ); 
		}
		
		// Forma de programacion con estructuras de control
		bool OR (boolean X, bool Y ){
		if (X || Y) {return true;} 
		else {return false;}
		}
		
		// Completar código
		bool XOR (bool X, bool Y ){
				; 
		}
		
		bool NAND (bool X, bool Y ){
				; 
		}		
		
		bool NOR (bool X, bool Y ){
				; 
		}
		\end{lstlisting}
}