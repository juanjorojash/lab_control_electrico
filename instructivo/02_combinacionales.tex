\chapter{Programación de funciones combinacionales.}

\section{\obj}
\capacidad
\begin{itemize}
    \item  Programar funciones booleanas combinacionales en Arduino.
    \item  Verificar la compuertas NAND y NOR son en si mismas un conjunto Universal, equivalentes a el conjunto de conectivas lógicas OR, AND y NOT.
    \item Verificar que a partir de los mintérminos de una expresión lógica se obtienen los circuitos con compuertas NAND y de los maxtérminos, los circuitos con compuertas NOR.
    \item Verificar que para obtener la solución mínima de un circuito lógico, es necesario sintetizar los circuitos tanto por mintérminos como por maxtérminos. 
    \item Programar multiplexores $4\times1$ y $8\times1$.
\end{itemize} 


\section{\mat}
\begin{itemize}
    \item 1 Arduino UNO o equivalente: MEGA o ESP32.
    \item 1 Multímetro.
    \item 10 Resistencias de 270 o 330 $\Omega$.
    \item 5 Resistencias de 1k$\Omega$.
    \item 10 interruptores pulsadores.
    \item 1 Protoboard.
    \item 10 Diodes emisor de luz (LEDs).
    \item 1 Computadora portátil.
\end{itemize} 

\section{Metodología}

Este laboratorio tiene una duración de 4 lecciones, repartidas en dos semanas. Los estudiantes deben mostrar durante las clases programadas las tres actividades propuestas. Deben recabar fotografías y resultados de los equipos de medición para elaborar las evidencias. Las evidencias se subirán al TecDigital la semana siguiente finalizadas las actividades.

\section{Práctica en Clase}

\subsection{Actividad 1}

Programe en Arruino una lógica combinacional que resuelva el siguiente problema:
\begin{itemize}
    \item En una planta industrial se desea automatizar un motor y una alarma de acuerdo a 4 sensores llamados $a$, $b$, $c$ y $d$ 
    \item La señal del motor se representa por la función lógica $f1$, y se activará  cuando los sensores $a$ y $b$ estén encendidos, o cuando los sensores $b$ y $d$ estén encendidos, o cuando los sensores $a$, $c$ y $d$ están encendidos.
    \begin{equation*}
        f1 = ab + bd + acd
    \end{equation*}
    \item La señal de  alarma se representa con la función lógica $f2$, y se activará siempre que estén todos los sensores apagados excepto $a$ o todos apagados excepto $c$ o estén encendidos $c$ y $d$ y el resto apagados. También la alarma se enciende cuando todos los sensores están encendidos o cuando están todos encendidos excepto $d$ o todos encendidos excepto $c$.
    \begin{equation*}
        f2 = a\bar{b}\bar{c}\bar{d} + \bar{a}\bar{b}c\bar{d} + \bar{a}\bar{b}cd + abcd + abc\bar{d} + ab\bar{c}d
    \end{equation*}
    \begin{equation*}
        f2 = a\bar{b}\bar{c}\bar{d} + \bar{a}\bar{b}c + abc + abd
    \end{equation*}
    \item Los pines $\left\lbrace 2,3,4,5\right\rbrace$ serán las entradas digitales $\left\lbrace a,b,c,d\right\rbrace$.
    \item Los pines $\left\lbrace 6,7,8\right\rbrace $ se reservan para $f1$. Se deben implementar tres funciones: la suma de productos $(SDP)$, el producto de sumas $(PDS)$ y la función resuelta por conectivas $NAND/NAND$. 
    \begin{align*}
        &f1_{SDP} = ab + bd + acd\\
        &f1_{PDS} = (a+b)\cdot (a+d)\cdot (b+d)\cdot (b+c)\\
        &f1_{NAND/NAND} = \overline{\overline{ab} \cdot \overline{bd} \cdot \overline{acd}}
    \end{align*}
    \item Los pines $\left\lbrace 9,10,11\right\rbrace $ se reservan para $f2$. Se deben implementar tres funciones: la suma de productos $(SDP)$, el producto de sumas $(PDS)$ y la función resuelta por conectivas $NOR/NOR$.
    \begin{align*}
        &f2_{SDP} = a\bar{b}\bar{c}\bar{d} + \bar{a}\bar{b}c + abc + abd\\
        &f2_{PDS} = (a+c)\cdot (a+\bar{b})\cdot(b+c+\bar{d})\cdot(\bar{b}+c+d)\cdot (\bar{a}+b+\bar{c})\\
        &f2_{NOR/NOR} = \overline{\overline{(a+c)}\cdot \overline{(a+\bar{b})}\cdot \overline{(b+c+\bar{d})} \cdot \overline{(\bar{b}+c+d)} \cdot \overline{(\bar{a}+b+\bar{c})}}
    \end{align*}
\end{itemize}
 
\subsubsection{Conteste las preguntas:}

¿Cómo sería el circuito digital de cada una de las funciones implementadas?
¿Como es la tabla de verdad experimental de cada salida versus la tabla teórica?


\subsection{Actividad 2}

Programe las funciones de un Multiplexor 2x1, 4x1 y 8x1.
Por otra parte, existe un circuito  combinacional que se comporta como la tabla de verdad mostrada en la Tabla \ref{tab:tv}.

\begin{table}[H]
\centering
\caption{Comportamiento esperado de la estructura lógica.}
\label{tab:tv}
\begin{tabular}{ccc}
    \toprule 
    $a$ & $b$ & $F(a,b)$ \\ 
    \midrule
    0 & 0 & $c + d$ \\ 
    0 & 1 & $\overline{c + d}$ \\ 
    1 & 0 & $\overline{c \cdot d}$ \\ 
    1 & 1 & $c \oplus d$ \\ 
    \bottomrule
\end{tabular} 

\end{table}

\subsubsection{Conteste las preguntas:}

¿El comportamiento de un multiplicador $8 \times 1$ coincide con la tabla teórica?
En la funcion Loop() , llame la función del MUX  $8 \times 1$ e ingrese parámetros $\left\lbrace a,b,c,d \right\rbrace $ de tal forma que se comporte igual que la tabla de verdad presentada en el Cuadro \ref{tab:tv}.¿Tiene el mismo comportamiento?
¿Es posible implementar la T.V. con dos MUX $4 \time 1$ y un MUX  $2 \time 1$?¿Como se implementa?
