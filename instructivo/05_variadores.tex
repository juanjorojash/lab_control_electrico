\chapter{Laboratorio:  Programación de variadores de frecuencia y arrancadores suaves.}

\section{\obj}
\capacidad
\begin{itemize}
	{\small
    \item  Alambrar el circuito de control y potencia del variador.
    \item  Programar un variador de frecuencia segun los requerimientos. 
	\item Alambrar el arrancador suave para un motor trifásico.
	\item Alambrar un arranque pare reversible usando un arrancador suave de un motor trifásico.
	\item Programar el variador de frecuencia.
	\item Medir el tiempo de las rampas de aceleración y desaceleración, y vincular con los parámetros programados.
 }
\end{itemize} 

 
\section{Equipos y materiales}
Para este laboratorio de necesitaran:
\begin{itemize}
	{\small \item 1 motor trifásico 1/2 hp de 3 puntas.
	\item 1 variador de frecuencia: \href{https://search.abb.com/library/Download.aspx?DocumentID=3AUA0000066143&LanguageCode=en&DocumentPartId=1&Action=Launch}{ABB ACS355} o \href{https://vfds.com/content/manuals/delta/delta-vfd-el-manual.pdf}{VFD-EL}
	\item 1 arrancador suave SIEMENS RW3013.
	\item 2 contactores.
	\item 1 botón pulsar cerrado.
	\item 2 botones pulsadores abiertos.
	\item 3 interruptores N.O.
	\item 1 amperimetro de gancho con medición INRUSH.
	\item 1 analizador de señales trifásico FLUKE 143B.
}
\end{itemize}

\section{Metodología}

Este laboratorio tiene una duración de 4 lecciones, repartidas en dos semanas. Los estudiantes deben mostrar durante las clases programadas las tres actividades propuestas. Deben recabar fotografías y resultados de los equipos de medición para elaborar las evidencias. Las evidencias se subirán al TecDigital la semana siguiente finalizadas las actividades.

\section{Práctica en Clase}

\subsection{Actividad 1}

Suponga que el arrancador suave es para un motor de una cinta transportadora que puede operar en ambos sentidos. Según el manual del \href{https://support.industry.siemens.com/dl/files/095/38752095/att_1310813/v1/Manual_softstarter_3RW30_3RW40_es-MX.pdf}{SIRIUS 3RW30} \cite{SIEMENS}, sección 13.1.2, seleccione el voltaje de inicio y la rampa de tiempo recomendada.
Implemente un circuito de control y potencia para que el motor arranque en dos sentidos, para esto puede usar como referencia la sección 16 de ejemplos. 
 
\subsubsection{Conteste las preguntas}

¿Puede mostrar el circuito de control del sistema?
Usando un analizador Fluke 143B  mida la corriente de arranque del motor.
¿Como son las señales de corriente y voltaje a la entrada del motor con/sin arrancador suave?
¿Cuanto tiempo tardó en aparecer el voltaje de línea en las terminales?¿Coincide con la rampa de tiempo pre-establecida?

\subsection{Actividad 2}
 

Utilizando el manual del variador, encuentre el código que re-inicial el VDF a sus valores de fábrica.
Luego ingrese los parámetros descritos en la sección 3.3 asociados al voltaje, corriente nominal y frecuencia base del motor.
Posteriormente seleccione un modo de entrada que permita operar el motor en ambos sentidos mediante cableado externo.
Programe rampas de aceleración entre las frecuencias mínimas y máximas de 20 segundos y de desaceleración de 30.
Programe siete frecuencias en el VDF, asumiendo torque constante en vacío,de tal forma que en el eje del motor se midan las velocidades de la Tabla \ref{Tab:velocidades}.

\begin{table}[H]
	\centering  
	\caption{Velocidades solicitadas}
	\label{Tab:velocidades}
	\begin{tabular}{|c|c|}
		\hline 
		Velocidad	&	Rev/min	\\	\hline 
		
		0	&	900	  	   	 \\ 
		1	&	1050	  	   	 \\ 
		2	&	1200	  	   	 \\ 
		3	&	1350	  	   	 \\ 
		4	&	1500	  	   	 \\ 
		5	&	1650	    	 \\ 
		6	&	1800        	 \\ 
		7	&	1950     	   	 \\ 
		\hline 
	\end{tabular} 
\end{table}
 

 
\subsubsection{Conteste las preguntas:}

Realice la gráfica de voltaje en función de la frecuencia ($V=\mathfrak{F}(f)$), ¿Es como se esperaba?
Realice una tabla con las mediciones de las velocidades medidas, ¿Cuál fue el porcentaje de error relativo de cada medición?
Si se programa el variador con frecuencias calculadas con torque variable, los errores mejoran o empeoran? ¿Por qué?
¿Son las gráficas PWM a la salida del variador son como se esperan? ¿Puede mostrar un gráfico? Si se desconecta una fase de entrada en VDF, ¿que sucedió con el VDF? Mida las rampas de aceleración y desaceleración, cumple lo programado? Cuanto tarda de pasar de la velocidad 3 a la 5 y viceversa, tienen estos tiempos sentido? Que pasaría si se programan rampas de 1 segundo de aceleración y des-aceleración?
 

%\section{Informe de evidencias para el TECDIGITAL}
%
% Para cada actividad copie los resultados que se imprimieron en el puerto serial. Adicionalmente, para las actividades 2 y 3 muestre las fotografías de los circuitos.
% 
% El informe que se sube al TEC DIGITAL debe ser un archivo PDF, con las siguiente partes:
% 
% \begin{enumerate}
% 	\item Identificación del laboratorio, Autores, fecha.
% 	\item Resumen
% 	\item Objetivos
% 	\item Descripción de la actividad x.
% 	\item Evidencia fotográfica de los circuitos. 
% 	\item Evidencia de los resultados obtenidos (gráficas, tablas, imágenes, etc ).
% 	\item Análisis de las preguntas planteadas en la actividad.
% 	\item Referencias.
% \end{enumerate}
% 
% Si el laboratorio posee x actividades los pasos 4,5,6 y 7 se repiten x veces. 



